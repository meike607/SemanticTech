\documentclass[12pt,a4paper]{article}

\usepackage[english]{babel}

\usepackage[left=20mm,right=20mm,top=25mm,bottom=20mm]{geometry}
\author{Jonas Harlacher}

\usepackage{fancyhdr}
\pagestyle{fancy}
\fancyhf{}
\rhead{Mat.Nr.: 57103, \today}

\begin{document}

{
\begin{center}
\begin{Large}
Outlining the content of ``Computing Machinery And Intelligence''
\end{Large}
\end{center}
In the paper ``Computing Machinery And Intelligence by A. M. Turing'' the question is asked, if machines can think, on which Turing iterates and describes how this could be achieved.
Instead of trying to answer the question he proposes to rephrase the question.
For this he describes a game with three players.
In this game there is an investigator who is trying to discern which of the other two players is male and female.
One of the two players is attempting to throw off the investigator and can lie to achieve this goal.
The other tries to help the investigator, possibly by telling the truth about who is the man and who is the woman.
The question now is phrased ``If the likelihood with which the investigator decides correctly is effected by a mashine taking the place of one of the players''.
The machine should take up the part of the man, and if the machine fares well, it should be safe to assume that the machine is able to think.
For this, the machine needs to be able to understand the questions and form answers, as a man would give them.

He also gives a quick overview on what a digital machine entails and how it can be modelled to act a certain way.
For this example he describes discrete state machines which allow a system to have multiple states between which it switches according to given instructions.
It is determined that a machine that is to compete in the imitation game should have sufficient memory and speed as a precondition to being judged, as well as a suitable programm.

In the paper at ``6.'' he outlines some possible arguments which stand against the idea that machines could ever think.
Certain arguments fail to show any proof for their claims, like the theological argument or the ``Arguments from various disabilities''.
Other arguments show the unwillingness to accept that a machine could think, such as the ``Argument from conciousness''.
It describes a statement from Professor Jefferson, which states that a machine can not think until it can feel.
Turing argues that Jefferson means that a machine needs to be able to understand what it says and not just rattle of its given instructions.
% To acknoledge this argument, means that the machine needs to be able to understand causation and correlation.
In ``(6) Lady Lovelace’s Objection'' a point is made that machines can only do the things we order them to do.
However this does not mean that a machine could not think of its own or that it can not create something new.
Turing elaborates on this and proposes that a mashine can be suitable programmed to posses the ability to originate ideas.

Later, in ``7.'', he describes how a mashine, with sufficient size in memory and instruction space and being programmed accordingly, could originate an entire theory given a certain stimuli.
The initial stimuli, will cause reactions within the mashines according to its programming.
By doing this the impulse will create a cascade of reactions which should arrive at a solution or a multitude of ideas.

He proposes a kind of supervised learning to train a child-like machine.
He hopes to recreate the steps that any real human would go through, where he starts with a mostly blank-slate machine, much like a child.
This machine is then taught by a teacher in some form with rewards and punishments to enforce correctness of the machine.
With this the problem of creating a thinking-mashine is divided into two parts: the initial programming and the learning process.
He elaborates on the difficulties that arise with teaching, such as rules that have to be defined by the teacher for the child to follow.
He describes how the process of modelling the machine and the teaching has to be repeated several times and decided how it should be adjusted in each iteration.
This is how he believes that a mashine can be trained to be able to play the imitation game and with that, to think.


}
\end{document}
