\documentclass[12pt,a4paper]{article}

\usepackage[english]{babel}

\usepackage[left=20mm,right=20mm,top=25mm,bottom=20mm]{geometry}
\author{Jonas Harlacher}

\usepackage{fancyhdr}
\pagestyle{fancy}
\fancyhf{}
\rhead{von Jonas Harlacher, \today}

\begin{document}

{
\begin{center}
\begin{Large}
A Critique of ``Computing Machinery And Intelligence''
\end{Large}
\end{center}
In his paper ``Computing Machinery And Intelligence'' Mr. Turing goes on about a few points that, from by perspective, are not helpful to decide the question
Such as elaborating on religious beliefs and E.S.P., which do not help to a point across.

He also fails to outline exactly how the machines are trained, he describes that a child machine is to be teached by reward and punishment.
However how the child is actually ``learning'' is not stated directly.
One would assume that a mechanism similar to the ``back-propagation'' in modern machine learning algorithms is to be used, to adjust the decision making process of the child-machine.

His Proposition to create an initial model that is trained like a child is much the same as it is in today's supervised learning algorithms.
He even dis

I do not believe that ``playing the imitation game'' and ``being able to think'' are equivalent. A machine that can think surely would have no problems to play the imitation game, however a machine that is designed to give human-like answers would also play the game well enough.
Since the questions are not equivalent is why I believe that Turing did not answer the question ``Can machines think?''.




}
\end{document}
