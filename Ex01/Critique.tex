\documentclass[12pt,a4paper]{article}

\usepackage[english]{babel}

\usepackage[left=20mm,right=20mm,top=25mm,bottom=20mm]{geometry}
\author{Jonas Harlacher}

\usepackage{fancyhdr}
\pagestyle{fancy}
\fancyhf{}
\rhead{Mat.Nr.: 57103, \today}

\begin{document}
{
\begin{center}
\begin{Large}
A Critique of ``Computing Machinery And Intelligence''
\end{Large}
\end{center}
In his paper ``Computing Machinery And Intelligence'' Mr. Turing wants to argue that a machine, that can fool the investigator in the imitation game, is capable of ``thinking''.

He describes that a child machine is to be teached by reward and punishment.
However how the child is actually ``learning'' is not stated directly.
One would assume that a mechanism similar to the ``back-propagation'' in modern machine learning algorithms is to be used, to adjust the decision making process of the child-machine accordingly.
His proposition to create an initial model that is trained like a child is much the same as it is in today's supervised learning algorithms.
The learning process is described as a teacher-student process, rather than a model that is being trained on data like it is today.

He also addresses the question how the rules of the mashine can change and claims the rules are of ephemeral validity.
I interpret this as, meaning that the mashine will adjust its decision making process according to experience.
With this the mashine can decide which rules are important to reach the right conclusion.

% I do not believe that ``playing the imitation game'' and ``being able to think'' are equivalent.
% A machine that can think surely would have no problems to play the imitation game, if taught correctly.
% However a machine that is designed to give human-like answers would also play the game well enough.
% Because the questions are not equivalent is why I believe that Turing did not answer the question ``Can machines think?''.

In all honesty I do not see where Mr. Turing fails in his conclusions.
To me it seems that he outlined a valid model of a learning mashine, albeit rather abstract.

}
\end{document}
